\section{Evaluation}

For this particular study we focus in on two different varieties of dried mango slices that we were donated by the study's patron:

\begin{itemize}
    \item Just Mango \\
        As the description says, this item is purely dried mango.
        As such, it is somewhat chewy and can get stuck in one's teeth.
        Moreover, we suspect that some amount of the sugars have left the item in the dehydration process, though the nutritional facts seem to claim otherwise.
    \item Soft and Juicy Mango \\
        This item includes additional sugars and preservatives to help maintain moisture in the item and counteract some of the chewy properties of the alternative.
        However, there are also fewer dietary fibers in this item, and thus may prove less effective as a viable snack over time.
        See Section~\ref{sec:futurework} for additional comments.
\end{itemize}

Table~\ref{tbl:eval} summarizes our results.

\begin{table*}[t]
    \centering
    \begin{tabular}{|l|l|l|l|}
        \hline
        Type                    & Taste               & Texture     & Overall \\
        \hline
        Just Mango              & Hint of sweetness   & Dry/Chewy   & Good \\
        \hline
        Soft and Juicy Mango    & Much more sweet     & Moist/Chewy & Very good \\
        \hline
    \end{tabular}
    \caption{Evaluation Results}
    \label{tbl:eval}
\end{table*}

Qualitatively, our subjects note that they were still tired and distracted from their true tasks at hand, and hence may invalidate the hypothesis that these snacks provide the necessary mental boost to stay engaged in the drudgery before them.
Further study is required.

Note also that the test subjects in this case were the authors themselves, thus no small amount of bias can be discounted.
